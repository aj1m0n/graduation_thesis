%%%%%%		next page 				%%%%%%
\newpage
\setcounter{chapter}{5}
\setcounter{section}{0}
%\setcounter{subsubsection}{0}


\begin{center}
\vspace{0.5cm}
\huge{\bf 第5章}
\par
\vspace{1cm}
\hrulefill
\par
\vspace{1cm}
\huge{\bf 結果と考察}
\par
\vspace{0.5cm}
\hrulefill
\vspace{1cm}
\par

%\normalsize
\begin{flushleft}
\large{{\bf 本章では,TScan故障解析結果と考察について述べる.}}
\end{flushleft}
\end{center}

\addcontentsline{toc}{chapter}{\protect\numberline{第5章}{結果と考察}}


\newpage
\section{故障解析結果}
本節では,故障解析ニューラルネットワークの出力別に結果を示す.ニューラルネットワークは各出力に対して,$0 \sim 1$までの数値を出力する.該当する出力の可能性が高いほど1に近づき,可能性が低い場合は0に近づく.

\newpage

\section{本章のまとめ}
本章では,センサネットワークの故障原因の解析として,ニューラルネットワークを用いた検証実験を行った結果を述べた.正常時のデータ,センサ故障のデータ,停電・バッテリ切れのデータについては精度高く解析することができた.一方で,電波障害データや下流ノード故障データの解析精度は高くはなかった.ニューラルネットワークの入力信号を細かく設定することで解析精度が向上させることが可能であることを示した.次章で本研究の結論について述べる.
\newpage
