%%%%%%%%%%%% ------------ chapter header ------------  %%%%%%%%%%%%
\chapterhead
% chapter title
{即興的及び人数チームプレイが可能な\\ブラウザネットワーキングゲーム基盤}

% chapter abst
{本章では,作成した即興的な多人数プレイが可能なブラウザネットワーキングゲームについて述べる.}


%%%%%%%%%%%% ------------ next page ------------  %%%%%%%%%%%%

\section{背景と関連研究}

近年,WebSocket\cite{websocket} によるリアルタイムな通信技術が注目されている.多くの技術職ではコミュニケーションツールを使い効率的に作業する需要がある.それを背景に,共同研究における WebSocket を用いた WebGL のリアルタイム同期手法の提案\cite{websocket_webgl}や Webデスクトップ共有の研究が行われている\cite{websocket_desktop}.さらに,スマートフォンの普及\cite{smartphone_share}を背景に,QR コード\cite{qrcode}の活用も浸透しており,QR コードリーダーの認知率は 9 割超であり,使用経験率は約 7 割であるという調査結果が出ている\cite{qrcoderesearch}.

我々は,スマートフォンをコントローラとしたコネクションの即時的な確立に着目した.そして即時参加が可能なブラウザアプリケーションの作成による,QR コードと WebSocket を用いたスマートフォンによる情報共有手法の提案をについて述べる\cite{ec2015}.

\section{システム概要}

ゲームプラットフォームの作成とともに一例として,多人数対応のブラウザシューティングゲームの作成をした.サーバーを起動してモニタとなる端末からブラウザでゲームページへアクセスすると\figref{qr}のようにQRコードが表示される.同じ画面で遊ぶプレイヤーはスマートフォン端末でQRコード読み取りをするとコントローラ用のURLへアクセスすることができプレイに参加が出来る.

コントローラはスマートフォンを横持ちで,シェイク(スマートフォンを振る)動作なども入力としてゲームに入れた\figref{controller}.socket通信\cite{webpagesocketio}を用いることでスマホで即時参加が可能でリアルタイムにプレイヤーの操作が出来る.

\myfig{qr.png}{認証QRコード}{qr}
% \myfigdouble{con.jpg}{help.png}{コントローラ,コントローラの説明}{controller}

\subsection{ゲームシステム}
ゲームの内容について説明する.

ゲームは平面のシューティングゲームを作成した.ステージ上をプレイヤー\figref{player}が移動できて壁で構成される部分はオブジェクトが通過できない\figref{stage}.

プレイヤーはHP(Hit Point),MP(Hit Point) を持っていて,ショット攻撃にMPを消費し,消費したMPはマップに散らばり取得するとMPが回復するという特徴のルールを加えた.

プレイヤーのアクションは移動とショット攻撃とダッシュの3つが行える.コントローラ右でショット,コントローラ左でショット攻撃,シェイクでダッシュが出来る.

プレイヤー数に対する処理速度の考察を行った.作成したゲームについて接続人数とFPSを取って回帰分析を行った\figref{graph}.
7台の接続時の28.28fpsで若干表示が重いる状態であった

\myfigtwo
{items.png}{プレイヤー・ショット}{player}
{game_play.png}{ゲームプレイ画面}{stage}
\myfighalf{g.png}{FPSグラフ}{graph}

%%%%%%%%%%%% ------------ next page ------------  %%%%%%%%%%%%

\newpage
\section{システム構成}
\subsection{構成図}
システムの構成は 2つを想定し,オンラインサーバを運用して動かす\figref{systemonline}に示す構成と,サーバを PC で立てることによりローカルネットワークのみでのプレイが出来る\figref{systemoffline}で示す構成を.

%  サーバ構成
\myfigtwo{systemonline.png}{システム構成図}{systemonline}
{systemoffline.png}{システム構成図2}{systemoffline}

\subsection{通信の流れ}
以下のような流れでコネクションを確立する\figref{connect_top}.
\begin{itemize}
    \item 1, 2. ディスプレイとなる端末からメインページ(ドキュメントルート/)にアクセスする.
    \item 3, 4. レスポンス時に socket コネクションを確立する.
\end{itemize}

ゲーム画面のプレイ準備までには以下の様な通信を行う\figref{connect_sub}.

\begin{itemize}
    \item 1, 2. クライアントがチーム選択ページ(/con)にアクセスする
    \item 3, 4, 5, 6. チームを選択完了ページ(con/team=)にアクセスし,socket のコネクションを確立する.
    \item 7 ディスプレイ端末や他のユーザへ追加情報を送信する.
\end{itemize}

ゲーム時の通信は\figref{node_game}に示すようにコントローラの入力をsocket を通してメインページを開いているクライアントへ送信し,プレイヤーのアクションへと同期している.

\myfig{node_top.png}{メインページ接続時の通信}{connect_top}
\myfig[width=12cm]{node_con.png}{コントローラ接続時の通信}{connect_sub}
\myfig[width=12cm]{node_game.png}{ゲーム時のコントロールクエリの同期}{node_game}

%%%%%%%%%%%% ------------ next page ------------  %%%%%%%%%%%%

\newpage

\section{本章のまとめ}
ゲームをプレイしている様子を\figref{game_main}に示す.
\myfig[width=12cm]{game_main.jpg}{プレイの様子}{play}
% 限界
今回作ったゲームは4人で接続で20秒に一度ほどゲームのフリーズが発生,8人での接続だと常時カクつきが見られた.ソケットで扱うデータが単純であれば性能の向上ができると考えられる.

% 解明点
本システムは本学のオープンキャンパスや学会でデモやポスターセッション発表を行った.参加者の多くは自身のスマートフォンでの参加ができたが,QR コードリーダーアプリが端末に入っていないケースが一割以下で発生した.
今後の発展としてゲームフレームワークなどを始点としてパッケージ化やネットワーク負荷の評価などを考えられる.

% 意義

% 展望
このシステムの応用としては,サーバーサイドの汎用化,ライブラリ化が望める.RaspberryPI への移植も行えれば, ルーターとなることでネットワーク環境のない被災地などでもモバイル端末への通信の可能性も広がると思われる.

\newpage
