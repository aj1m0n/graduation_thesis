%%%%%%		next page 				%%%%%%
\newpage
\setcounter{chapter}{3}
\setcounter{section}{0}
%\setcounter{subsubsection}{0}

\begin{center}
\vspace{0.5cm}
\huge{\bf 第3章}
\par
\vspace{1cm}
\hrulefill
\par
\vspace{1cm}
\huge{\bf 即興的及び人数チームプレイが可能なブラウザネットワーキングゲーム機版}
\par
\vspace{0.5cm}
\hrulefill
\vspace{1cm}
\par

%\normalsize
\begin{flushleft}
\large{{\bf TODO: 本章では,本研究対象としている群馬県館林市に構築したTScan:微気象センサネットワークについて述べる.}}
\end{flushleft}
\end{center}

\addcontentsline{toc}{chapter}{\protect\numberline{第3章}{即興的及び人数チームプレイが可能なブラウザネットワーキングゲーム機版築}}

\newpage
\section{システム概要}

%%%%%%		next page 				%%%%%%
\newpage

\section{本章のまとめ}
TODO: 本章では,群馬県館林市に設置したTScan:微気象センサネットワーク構築について述べ,TScanで使用しているリーフセンサノード,アクセスポイントセンサノード,シンクシステム,TScanサーバについて仕様を述べた.また,各センサデバイスにおける電源確保などの設置上の制約を明らかにした.次章では,故障解析に使用するニューラルネットワークならびに,TScan:微気象センサネットワークにおける故障解析手法について述べる.

\newpage
