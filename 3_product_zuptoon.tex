%%%%%%		next page 				%%%%%%
\newpage
\setcounter{chapter}{3}
\setcounter{section}{0}
%\setcounter{subsubsection}{0}

\begin{center}
\vspace{0.5cm}
\huge{\bf 第3章}
\par
\vspace{1cm}
\hrulefill
\par
\vspace{1cm}
\huge{\bf 即興的及び人数チームプレイが可能なブラウザネットワーキングゲーム基盤}
\par
\vspace{0.5cm}
\hrulefill
\vspace{1cm}
\par

%\normalsize
\begin{flushleft}
\large{{\bf 本性では, 作成した即興的な多人数プレイが可能なブラウザネットワーキングゲームについて述べる.}}
\end{flushleft}
\end{center}

\addcontentsline{toc}{chapter}{\protect\numberline{第3章}{即興的及び人数チームプレイが可能なブラウザネットワーキングゲーム機版築}}

\newpage
\section{システム概要}
TODO:\\


% ゲームプラットフォームの作成とともに一例として,多人数対応のブラウザシューティングゲームを作成をした(\figref{fig:play}).サーバーを起動してモニタとなる端末からブラウザでゲームページへアクセスすると(\figref{fig:qr})のようにQRコードが表示される.同じ画面で遊ぶプレイヤーはスマートフォン端末でQRコード読み取りをするとコントローラ用のURLへアクセスすることができプレイに参加が出来る.
% 
% コントローラはスマートフォンを横持ちで,シェイク(スマートフォンを振る)動作なども入力としてゲームに入れた(\figref{fig:controller}).socket通信を用いることでスマホで即時参加が可能でリアルタイムにプレイヤーの操作が出来る.

\myfig{node_con.png}{コントローラ接続時の通信}{connect_sub}

%%%%%%		next page 				%%%%%%
\newpage

\section{本章のまとめ}
TODO: 本章では,群馬県館林市に設置したTScan:微気象センサネットワーク構築について述べ,TScanで使用しているリーフセンサノード,アクセスポイントセンサノード,シンクシステム,TScanサーバについて仕様を述べた.また,各センサデバイスにおける電源確保などの設置上の制約を明らかにした.次章では,故障解析に使用するニューラルネットワークならびに,TScan:微気象センサネットワークにおける故障解析手法について述べる.

\newpage
