%%%%%%%%%%%% ------------ chapter header ------------  %%%%%%%%%%%%
\chapterhead
% chapter title
{序論}
% chapter abstraction
{本章では,本研究の背景と目的および,本論文の内容構成について述べる.}
%%%%%%%%%%%% ------------ next page ------------  %%%%%%%%%%%%

\section{背景}
Twitter に代表されるマイクロブログの広まりやスマートフォンの普及を背景に,ソーシャルメディアに人々の自発的で自然な反応が多く含まれるようになり,容易に取得ことが可能になった.

\section{本研究の目的}
本研究においても,リアルタイムの個別ユーザからの実世界のイベントに関する反応を入手し,トレンド分析やイベントの整理,共有などを行う様々なサービスを最新のサーバ技術を用いて構築した.リアルタイムな情報共有に注目し,複数のアプリケーションの作成を通し共通基盤を構築したことについて述べる.

\section{本論文の構成}
本論文の以下の構成は次のようになっている.\\
第2章では,マイクロブログを用いた経路やイベントの検出と可視化について述べる.\\
第3章では,即興的なブラウザ通信のシステムを提案について述べる.\\
第4章では,Twitter のローカルトレンドの抽出について述べる.\\
第5章では,モバイル端末センシングサーバAPIの構築について述べる.\\
第6章では,GPS 経路ノイズ除去手法について述べる.\\
最後に,第8章で本論文の結論を述べる.\\
