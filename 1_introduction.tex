%%%%%%%%%%%% ------------ chapter header ------------  %%%%%%%%%%%%
\chapterhead
% chapter title
{序論}
% chapter abstraction
{本章では,本研究の背景と目的および,本論文の内容構成について述べる.}
%%%%%%%%%%%% ------------ next page ------------  %%%%%%%%%%%%

\section{背景}
交通工学はエッジセンサから得られたデータを活用することで,
可能性が大きく広がる分野である.以前より,速度違反検知,ナンバープレート検知,渋滞検知などは日本のみならず先進国で導入されている.
それらのセンサは大規模かつ高額になることが多く,交通量の多い交差点や高速道路に限定して設置されていることが多い.
そのため,これらのセンサを活用した交通最適化を行っている事例は筆者の知る限り無い.
また,住宅街や二車線道路などでは,これらのセンサの設置がまだ進んでいないのが現状である.
これらの道路での渋滞は信号機の制御の最適化を行うことで解消すると考えられており,
我々は次世代の動的な信号機の開発を行なうことにした.
動的な信号機とは,交差点付近の交通状況に応じて最適な信号の制御を行うことができる信号機のことである.
この信号機の開発にあたり,交差点付近の交通状況を監視するセンサを開発する必要がある.
そこで,我々は監視カメラと画像処理を用いた信号機のセンサを開発した.
画像処理の中でも画像認識とリアルタイムトラッキングを用いることで,各車両の状態はもちろんのこと,交差点付近にいる人の状態も監視することができると考えられる.%kai

\section{本研究の目的}
本研究は信号機の動的な制御を行うために,センサとして画像処理を用いて交差点付近の車両の状態を推定することを目的としている.
認識する項目は,車両の位置情報,車両の速度,(車両の種類,車両の軌跡,交差点内の車両の右左折待ち)である.%kai
これらのデータを用いることで円滑な交通に繋がると同時に,これまで取ることのできなかった交通状況のデータを取ることが可能になる.

\section{本論文の構成}
本稿では,2章で関連研究について述べ,3章で提案システムの概要について説明し,4章ではアルゴリズムについて説明をする.5章では結果,6章では考察を述べる.最後に7章で今後の展望およびまとめを述べる.


\newpage
