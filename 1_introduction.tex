%%%%%%%%%%%% ------------ chapter header ------------  %%%%%%%%%%%%
\chapterhead
% chapter number, title
{1}{序論}
% chapter abst
{本章では,本研究の背景と目的および,本論文の内容構成について述べる.}
%%%%%%%%%%%% ------------ next page ------------  %%%%%%%%%%%%

\section{背景}
この世界は,リアルタイムな情報共有が求められる.\\
マイクロブログと位置情報から得られたら便利.\\
HTML5 は発展していてモバイルでも便利.\\


%%%%%%%%%%%% ------------ next page ------------  %%%%%%%%%%%%
\newpage

\section{本研究の目的}
SNS 使っていろいろWebアプリケーションとしてやってみるよ.
即興的な情報共有システムの実現をしてみる.

%%%%%%%%%%%% ------------ next page ------------  %%%%%%%%%%%%
\newpage

\section{本論文の構成}
本論文の以下の構成は次のようになっている.\\
第2章では,本論文で使用する諸概念について述べる.\\
第3章では,即興的なブラウザ通信のシステムを提案し.\\
第4章では,マイクロブログを用いた経路やイベントの検出と可視化について述べる.\\
最後に,第6章で本論文の結論を述べる.\\
