%%%%%%%%%%%% ------------ chapter header ------------  %%%%%%%%%%%%
\chapterhead
% chapter title
{序論}
% chapter abstraction
{本章では,本研究の背景と目的および,本論文の内容構成について述べる.}
%%%%%%%%%%%% ------------ next page ------------  %%%%%%%%%%%%

\section{背景}
交通工学はエッジセンサから得られたデータを活用することで,
可能性が大きく広がる分野である.以前より,速度違反検知,ナンバープレート検知,渋滞検知などは日本のみならず先進国で導入されている.
それらのセンサは大規模かつ高額になることが多く,交通量の多い交差点や高速道路に限定して設置されていることが多い.
そのため,これらのセンサを活用した交通最適化を行っている事例は筆者の知る限り無い.
また,住宅街や二車線道路などでは,これらのセンサの設置がまだ進んでいないのが現状である.
これらの道路での渋滞は信号機の制御の最適化を行うことで解消すると考えられており,
我々は次世代の動的な信号機の開発を行なうことにした.
動的な信号機とは,交差点付近の交通状況に応じて最適な信号の制御を行うことができる信号機のことである.
この信号機の開発にあたり,交差点付近の交通状況を監視するセンサを開発する必要がある.
そこで,我々は監視カメラと画像処理を用いた信号機のセンサを開発した.
画像処理の中でも画像認識とリアルタイムトラッキングを用いることで,各車両の状態はもちろんのこと,交差点付近にいる人の状態も監視することができると考えられる.%kai

\subsection{関連研究}
IoT(Internet of Things)の普及に伴い,さまざまな場所でのセンシングやデータの収集が可能となった.
交通工学の分野でも例外ではなく,古くは車両データを扱った技術として速度違反自動取締装置(オービス)が挙げられる.
また,近年ではITS(Intelligent Transport Systems:高度道路交通システム)
の観点からさらに高度で詳細な交通データを扱う研究やプロジェクトが注目されている.
IEEE/CVF Conference on Computer Vision and Pattern Recognition (CVPR) Workshopsでは
AI City Challenge\cite{Naphade_2020_CVPR_Workshops}と称し,ITSに関するワークショップを毎年開催している.
ここでは,カメラによる車両の識別,スピード検知,ナンバープレートの読み込みによるマルチカメラでの車両の再識別などのタスクが用意されており,
国際的に見てもITSが注目されていることがわかる.

しかし,現状としてそれらの装置が一般的な道路や交差点に取り付けられていることは少なく,高速道路や交通量の多い幹線道路に限定し取り付けれていることが多い.
これは,装置自体が高価で大規模化しやすいことが原因であると考えられる.
さらに,日本においてはそれらのデータは,違反や犯罪,渋滞の検知として用いられていることが多いものの,
交通の最適化という観点からデータを活用している事例は少ない.
これは,データの質の低さ,ラベルの欠如,データの利権などが障壁になっていることや
そもそも最先端技術を一般道路に活用しようとする働きかけが少ないことが考えられる.
しかし,働きかけは技術の進歩とともに徐々に増えていくものであり事実,〇〇 %1
やトヨタモビリティなど
次世代のインフラについて考える機会が増えつつある.
インフラの中でも重要な要素の一つに信号機がある.信号機の制御は,定時に切り替わる方式やランダムな時間間隔で切り替わる方式,押しボタン式,感応式など様々な制御方法が存在する.
しかし,それぞれの信号機が最適機であるとは限らない.信号機には
,車両の位置情報,車両の速度,(車両の種類,車両の軌跡,交差点内の車両の右左折待ち)
このことから本研究では,

こうしたカメラ画像から複数のオブジェクトを検出し追跡するタスクをコンピュータビジョンや機械学習分野ではMOT(Multi Object Tracking)と呼ぶ.
MOTでは,リアルタイムか否かで大きく二つにタスクを分けることができる.
リアルタイムではない処理としてTNT(TrackletNet Tracker)は,2次元のカメラ映像からオブジェクトを検出し,オブジェクトの軌跡を深層学習
を用いて推定していく方法である.\cite{wang2019exploit} \cite{tang2018single} TNTは,カメラ映像内で複雑かつ大量のオブジェクトを追跡することに向いているものの,カメラ映像を全て読み込んで処理するため,リアルタイム性
はなく,本研究では用いることができない.
一方でリアルタイムでの処理として,SORT(Simple Online and Realtime Tracking)がある.\cite{bewley2016simple} \cite{wojke2017simple}
これは,検出されたオブジェクトの座標をカルマンフィルタ\cite{kalman1960new}を用いて,リアルタイムでのオブジェクトの追跡を可能としている.
前述で述べたTNTと比べると複雑なMOTを行うことは難しいものの,ある程度規則性を持った車両などのMOTはリアルタイムで処理することが可能である.

\section{本研究の目的}
本研究は信号機の動的な制御を行うために,センサとして画像処理を用いて交差点付近の車両の状態を推定することを目的としている.
認識する項目は,車両の位置情報,車両の速度,(車両の種類,車両の軌跡,交差点内の車両の右左折待ち)である.%kai
これらのデータを用いることで円滑な交通に繋がると同時に,これまで取ることのできなかった交通状況のデータを取ることが可能になる.

\section{本論文の構成}
本稿では,2章で関連研究について述べ,3章で提案システムの概要について説明し,4章ではアルゴリズムについて説明をする.5章では結果,6章では考察を述べる.最後に7章で今後の展望およびまとめを述べる.


\newpage
