%%%%%%%%%%%% ------------ chapter header ------------  %%%%%%%%%%%%
\chapterhead
% chapter title
{序論}
% chapter abstraction
{本章では,本研究の背景と目的および,本論文の内容構成について述べる.}
%%%%%%%%%%%% ------------ next page ------------  %%%%%%%%%%%%

\section{背景}
交通工学はエッジセンサから得られたデータを活用することで,
可能性が大きく広がる分野である.速度違反検知,ナンバープレート検知,渋滞検知などは日本のみならず先進国で導入されている.
しかしながら,それらのデータを交通の最適化に活用している例はまだ少ない.
これは,データの質の低さ,ラベルの欠如,データの利権などが障壁になっていることが考えらえる.
またそれらのセンサは大規模かつ高額になることが多く,交通量の多い交差点や高速道路に限定して設置されていることが多い.
したがって,我々が普段使う道路や交差点ではこれらの導入は遅れていることが現状である.そこで,我々はオープンソースデータと安価なエッジセンサを活用し,
交差点付近の車両の状態を推定することができるデバイスと,信号機を最適化するシステムを開発した.
これにより今まで取ることができなかったその地点の詳細な車両のデータの取得と動的な信号機の信号操作を可能にした.


\section{本研究の目的}
本研究においても,リアルタイムの個別ユーザからの実世界のイベントに関する反応を入手し,トレンド分析やイベントの整理,共有などを行う様々なサービスを最新のサーバ技術を用いて構築した.リアルタイムな情報共有に注目し,複数のアプリケーションの作成を通し共通基盤を構築したことについて述べる.

\section{本論文の構成}
本稿では,2章で関連研究について述べ,3章で提案システムの概要について説明し,4章ではアルゴリズムについて説明をする.5章では結果,6章では考察を述べる.最後に7章で今後の展望およびまとめを述べる.


\newpage
