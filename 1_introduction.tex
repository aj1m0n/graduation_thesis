%%%%%%%%%%%% ------------ chapter header ------------  %%%%%%%%%%%%
\chapterhead
% chapter title
{Introduction}
% chapter abstraction
%% {本章では,本研究の背景と目的および,本論文の内容構成について述べる.}
{In this chapter, the background of this study and the purpose and describes the contents of the configuration of the present paper.}
%%%%%%%%%%%% ------------ next page ------------  %%%%%%%%%%%%

%% \section{背景}
%% Twitter に代表されるマイクロブログの広まりやスマートフォンの普及を背景に,ソーシャルメディアに人々の自発的で自然な反応が多く含まれるようになり,容易に取得ことが可能になった.

\section{Background}
Against the background of the spread and smartphones spread of micro-blog, which is represented by Twitter, it is as spontaneous and natural reaction of people is contained in a large amount in social media, has become easily able to acquire it.


%% \section{本研究の目的}
%% 本研究においても,リアルタイムの個別ユーザからの実世界のイベントに関する反応を入手し,トレンド分析やイベントの整理,共有などを行う様々なサービスを最新のサーバ技術を用いて構築した.リアルタイムな情報共有に注目し,複数のアプリケーションの作成を通し共通基盤を構築したことについて述べる.

%% \section{本論文の構成}
%% 本論文の以下の構成は次のようになっている.\\
%% 第2章では,マイクロブログを用いた経路やイベントの検出と可視化について述べる.\\
%% 第3章では,即興的なブラウザ通信のシステムを提案について述べる.\\
%% 第4章では,Twitter のローカルトレンドの抽出について述べる.\\
%% 第5章では,モバイル端末センシングサーバAPIの構築について述べる.\\
%% 第6章では,GPS 経路ノイズ除去手法について述べる.\\
%% 最後に,第8章で本論文の結論を述べる.\\

\section{Purpose}
In the present study, to obtain a reaction related to real-world events from the real-time of the individual user, organization of trend analysis and events, to construct a variety of services to carry out, such as shared by using the latest server technology. Focusing on real-time information sharing, we describe that it has built a common infrastructure through the creation of more than one application.

% \section{The configuration of the present paper}

The following configuration of the present paper is organized as follows.Chapter 2 describes the detection and visualization of routes and events using a micro blog.Chapter 3 describes the proposed system of improvised browser communication.Chapter 4, describes the extraction of local trend on Twitter.In Chapter 5, describes the construction of a mobile terminal sensing server API.In Chapter 6, we describe the GPS route noise removal technique.Finally, we describe the conclusion of this paper in Chapter 8.

