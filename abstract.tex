%%%%%%		2 page 				%%%%%%
\newpage
\pagestyle{plain}
\pagenumbering{roman}
\begin{center}
\LARGE{学士論文要旨 \hspace{10mm} 2015年度(平成27年度)}\\

\vspace{10mm}

\LARGE{マイクロブログの解析を応用したソーシャル分析}\\
\end{center}

\begin{center}
概要\\
\end{center}

 環境測定を目的とした無線マルチホップセンサネットワークが提唱されて10年以上経過しており,多くの実験が行われてきた.しかし多くの場合大学内や私有地をはじめとした限定された環境内での実験であり,実際に市街地に展開した事例は多くない.筆者らは2009年7月より2年間の連続運用を目指し,群馬県館林市にIEEE802.15.4 無線マルチホップ微気象センサネットワ ークを構築した.18ヶ月間の連続運用を行う中で,多くの無線マルチホップネットワークの障害やセンサノードの故障が発生している.本研究では,連続運用中に得られた故障データに基づき,計画・設置・運用各段階における故障原因をニューラルネットワークにより解析する手法を構築し,今後同様のシステムを構築する際の指針を提案した.

\begin{flushleft}キーワード:\\
\end{flushleft}
{\underline{ブラウザネットワーキング}, \underline{マイクロブログ}, \underline{人流} ,\underline{感情解析} }


\begin{center}
\vspace{10mm}
\begin{flushright}\large 東京電機大学大学院 未来科学研究科 情報メディア学専攻\\
\LARGE 高橋 洸人\\
\end{flushright}
\end{center}

%%%%%%		3 page 				%%%%%%
\newpage

\begin{center}
\LARGE{Bachelor's Thesis Academic Year 2015}\\

\vspace{10mm}
\LARGE{The Use of Neural Networks for Fault Analysis in Urban Sensor Networks}\\
\end{center}
\begin{center}
Abstract\\
\end{center}
The wireless multi-hop sensor network for environment monitoring was proposed more than ten years ago. However, the most systems were investigated in a privately-controlled community and few systems were examined in the local-government urban environment.In this study, we installed an IEEE 802.15.4 wireless multi-hop sensor network to Tatebayashi City, Gumma Prefecture, where anti-hyperthermia is focused on. The installed system is planned to be in operation until mid 2011. In this paper, we describe our experience and acquired knowledge in planning, installation, and operation and advices for the future planning.
%Technologies for recognizing things using short-range wireless communications are becoming available. However, the proposed systems assume a centralized server. We propose a system without a server called RW-Link in which tags pertained to things have links to each other. In RW-Link things are classified into fixed and mobile things where a mobile thing is associated to the nearest fixed thing, thereby enabling the location of a specified thing. To reduce power consumption of a mobile thing, we introduce a mechanizms: location registration triggered by movement. In this paper we described the design and the implementation of RW-Link.
\\\\

\begin{flushleft}Keyword:\\
\end{flushleft}
{\underline{Micro Climate},\underline{Sensor Network},\underline{Neural Network}}

\begin{flushright}
\vspace{10mm}

\vspace{5mm}
\large Department of Information and Media Engineering,\\
Tokyo Denki University\\
\begin{flushright}\LARGE Hiroto TAKAHASHI\\
\end{flushright}

\end{flushright}
