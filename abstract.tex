%%%%%%%%%%%% ------------ 2 page ------------  %%%%%%%%%%%%
\newpage
\pagestyle{plain}
\pagenumbering{roman}
\begin{center}
\LARGE{学士論文要旨 \hspace{10mm} 2015年度(平成27年度)}\\

\vspace{10mm}

\LARGE{即時性の伴うイベントを可視化・共有するWebサービス群}\\
\end{center}

\begin{center}
概要\\
\end{center}

TODO: Twitterをデータとした人流の予測, イベントの検知
<\ >近年ネットゲームによるネットワーク負荷が懸念されている.震災時の避難所生活ではネットワークが確立できないという問題が露呈した.
我々はインターネット上のサーバーを必要としない即興的なクライアント間のリアルタイム通信を実現する,ブラウザネットワーキングを利用したゲーミング基盤の構築した.
<\ >その場で自分のスマートフォン端末を用いて多人数で同時にプレイの出来るゲームを実現する.

\begin{flushleft}キーワード:\\
\end{flushleft}
{\underline{ブラウザネットワーキング}, \underline{マイクロブログ}, \underline{人流}}


\begin{center}
\vspace{10mm}
\begin{flushright}\large 東京電機大学大学院 未来科学研究科 情報メディア学専攻\\
\LARGE 高橋 洸人\\
\end{flushright}
\end{center}

%%%%%%%%%%%% ------------ 3 page ------------  %%%%%%%%%%%%
\newpage

\begin{center}
\LARGE{Bachelor's Thesis Academic Year 2015}\\

\vspace{10mm}
\LARGE{Web services that the immediacy of the associated event to visualize and share}\\
\end{center}
\begin{center}
Abstract\\
\end{center}

TODO: 
<\ >Network load due to net game is concerned. In the shelter life at the time of the earthquake was exposed is a problem that the network can not be established in recent years.
<\ >We realize the real-time communication between the improvised clients that do not require a server on the Internet, and the construction of the gaming platform that utilizes the browser networking.
<\ >To realize the play of the game can be at the same time by many people using their smartphone terminal on the spot.
\\\\

\begin{flushleft}Keyword:\\
\end{flushleft}
{\underline{Browser Networking},\underline{Microblogging},\underline{People Flow}}
% {\underline{ブラウザネットワーキング}, \underline{マイクロブログ}, \underline{人流}}

\begin{flushright}
\vspace{10mm}

\vspace{5mm}
\large Department of Information and Media Engineering,\\
Tokyo Denki University\\
% \large 東京電機大学大学院 未来科学研究科 情報メディア学専攻\\
\begin{flushright}\LARGE Hiroto TAKAHASHI\\
\end{flushright}

\end{flushright}
