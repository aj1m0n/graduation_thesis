%%%%%%%%%%%% ------------ chapter header ------------  %%%%%%%%%%%%
\chapterhead
% chapter title
{GPS 経路ノイズ除去}
% chapter abst
{本章では GPS のデータのログにおけるノイズフィルタリング手法の提案について述べる.}

%%%%%%%%%%%% ------------ next page ------------  %%%%%%%%%%%%

\section{背景と関連研究}
GPS 技術は実時間の高精度な測位を可能にした技術であり,多様な分野で活用されてる.一方で GPS による測位誤差についての問題も広く知られている\cite{gps_wrong}.

\section{サンプルデータ}
実験には移動距離が長く,特殊な動きをするスキーで記録したデータをサンプルとして用いた.\figref{ski_days}3日間のスキーデータのうち,2日目のデータが特にノイズの見られた.5秒間隔でログを記録した GPSデータ 1963個を使用する.
サンプルデータらからは\figref{gps_noise} のような GPS のノイズが見られた.

% 
\myfigtwo{ski_days.png}{スキーでの取得データ}{ski_days}
{gps_noise.png}{GPS のノイズ}{gps_noise}









\section{アルゴリズム}
データは一定間隔で記録しているので,連続したデータの 2 点間の距離の大きさから速度が求まる.明らかに不自然な速度で移動したとみなされるデータの削除を行う.
手順は\figref{sequens_filter}のように行う.また,距離の計算にはヒュベニの公式を用いた.
2点間距離の降順に上位の割合をしきい値としてマップにプロットし,評価を行った

\myfig[width=10cm]{sequens_filter.png}{提案フィルタリング手順}{sequens_filter}






\section{実験結果}
30\%,10\%, 2\% を誤差データの割合としてフィルタリングした結果を\figref{filtered} と \figref{filtered2} に示す.
特に大きな誤検知データの除去は行うことが出来た.しかし,\figref{gps_inroom} に示すような屋内での GPS ログの誤差はカバーできなかった.

\myfig[width=7cm]{filtered.png}{フィルタ後のデータ,青: フィルタ前,水色: フィルタ後}{filtered}
\myfig[width=7cm]{filtered2.png}{フィルタ後のデータ2,黄緑: 30\%,水色: 10\%, 赤: 2\%}{filtered2}
\myfig[width=7cm]{gps_inroom.png}{屋内でのデータフィルタ結果,青: フィルタ前,赤: 30\%, 黄緑: 10\%}{gps_inroom}




\section{本章のまとめ}
しきい値をノイズデータの割合として定める事によるフィルタリング手法を提案した.GPS 情報には高度,や精度の情報が付加されているためそれを踏まえた手法の改善などが考えられる.また,フィルタリング処理によりサンプルのデータ数が減ってしまうのは改善点であり,データクレンジング処理も考案が必要である.
% 限界
% 解明点
% 意義
% 展望


\newpage
