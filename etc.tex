%%%%%%		next page 				%%%%%%
\newpage

\begin{flushleft}
{\huge{\bf 謝辞}}\\
\vspace{1cm}
 本研究を進めるにあたり,研究指導をはじめあらゆる面でご協力して下さった東京電機大学未来科学部情報メディア学科戸辺義人教授に深く感謝致します.
\par
 最後に研究の日々を共にした,東京電機大学未来科学部情報メディア学科ユビキタスネットワーキング研究室の友人達に感謝の意を表します.\\

\vspace{3cm}
\begin{flushright}
2011年3月31日\\
高木 篤大\\
\end{flushright}
\end{flushleft}




%%%%%%		next page 				%%%%%%
\newpage



\begin{flushleft}
{\huge{\bf 学外発表}}\\
\vspace{1cm}
\begin{enumerate}
	
\item \underline{高木 篤大},菅生 啓示,石田 泰之,森田 達也,岩本 健嗣,蔵田 英之,戸辺 義人, "微気象ネットワークセンシングの実際:群馬県館林市の例", 電子情報通信学会技術研究報告. USN, ユビキタス・センサネットワーク109(248), 13-18, 2009年10月.
\item \underline{高木 篤大},木實新一,戸辺 義人, "TScan:微気象ネットワーク構築の実際", The 6th IEEE Tokyo Young Reasearchers Workshop, pp.26, 2009年12月.
\item \underline{高木篤大},菅生啓示,岩本健嗣,木實新一,小笠原拓也,蔵田英之,戸辺義人, "TScan:微気象センサネットワークの構築",情報処理学会 第72回全国大会, 第3分冊, pp.289-290, 2010年3月. 


%\item \underline{Ryutaro Nakata}, "Disributed Links of Mono", Asian Workshop on Ubiquitous and Embedded Computing (AWUEC), Taiwan, August, 2008.

%\item \underline{中田 龍太郎,石塚 宏紀,岩井 将行,戸辺 義人}, "実世界リンクシステムためのGUIの設計", 情報処理学会 FIT2008 第7回情報科学技術フォーラム,pp.269-270, 2008年9月.

%\item \underline{中田 龍太郎,石塚 宏紀,岩井 将行,テープウィロージャナポン ニワット,
%戸辺 義人}, "加速度センサを用いた移動トリガ位置登録システムの設計と実装", FIT2009 %第8回情報科学技術フォーラム,第4分冊,pp.263-264, 2009年9月.


\end{enumerate}
\end{flushleft}


%%%%%%		next page 				%%%%%%
\newpage

\renewcommand{\bibname}{参考文献}

\begin{thebibliography}{1}
\bibitem{tanita}
からだカルテ,http://www.karadakarute.jp/tanita/
\bibitem{field}
フィールドサーバ,http://model.job.affrc.go.jp/FieldServer/default.htm
\bibitem{wini}
ウェザーニュース,http://weathernews.jp/
\bibitem{roofnet}
John Bicket,Daniel Aguayo,Sanjit Biswas,Robert Morris,"Architecture and Evaluation of an Unplanned 802.11b Mesh Network",The 11th Annual International Conference on Mobile Computing and Networking(ACM Mobicom 2005),Aug 2005.
\bibitem{sensorandrew}
Anthony Rowe,Mario Berges,Gaurav Bhatia,Ethan Goldman,Ragunathan (Raj) Rajkumar,Lucio Soibelman,James Garrett,Jose M. F. Moura,"The Sensor Andrew infrastructure for large-scale campus-wide sensing and actuation",Information Processing in Sensor Networks 2009(IPSN 2009),Oct 2009. 
\bibitem{orbs}
Lufeng Mo,Yuan He,Yunhao Liu,Jizhong Zhao,Shaojie Tang,Xiang-Yang Li,Guojun Dai,"Canopy Closure Estimates with GreenOrbs: Sustainable Sensing in the Forest", The 7th ACM Conference on Embedded Networked Sensor Systems(ACM SenSys 2009),Nov 2009.
\bibitem{amedas}
気象庁 地域気象観測システム(AMeDAS),\\ http://www.jma.go.jp/jma/kishou/know/amedas/kaisetsu.html
\bibitem{gerira}
ウェザーニュース ゲリラCh.,http://weathernews.jp/guerrilla/
\bibitem{nettyu}
ウェザーニュース 熱中症アラート,http://weathernews.com/jp/c/press/2007/070810.html
\bibitem{smartroom}
Smart Rooms,http://vismod.media.mit.edu/vismod/demos/smartroom/
\bibitem{smartspace}
Tokuda Hidetoshi,Takahashi Gen,Kadota Masaya,Iwai Masayuki,Tokuda Hideyuki,"u-Con: A Smart Space Remote Control System",情報処理学会研究報告 UBI 2005,2005年6月.
\bibitem{mitsubishi}
高田 憲一,田村 智只,梶 孝則,石橋 孝一,矢野 雅嗣,"ユビキタス・ワイヤレス・センサ・ネットワークにおける故障ノード特定手法の一検討",電子情報通信学会技術研究報告 IN,2006.
\bibitem{citysense}
Rohan Narayana Murty, Abhimanyu Gosain, Matthew Tierney,Andrew Brody,Amal Fahad,Josh Bers,Matt Welsh,"CitySense: A Vision for an Urban-Scale Wireless Networking Testbed",Technical Report Harvard University,2007.
\bibitem{airynotes}
Masaki Ito,Yukiko Katagiri,Mikiko Ishikawa,Hideyuki Tokuda,"Airy Notes: An Experiment of Microclimate Monitoring in Shinjuku Gyoen Garden",4th International Conference on Networked Sensing Systems(INSS 2007),Jun 2007.
\bibitem{dpsn}
M.M.H. Khan,H.K. Le,M. LeMay,P. Moinzadeh,L. Wang,Y. Yang,D.K. Noh,T.F.Abdelzaher,C.A. Gunter,J. Han,X. Jin,"Diagnostic Powertracing for Sensor Node Failure Analysis",The 9th International Conference on Information Processing in Sensor Networks(IPSN 2010),Apr 2010.
\end{thebibliography}
