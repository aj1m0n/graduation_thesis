%%%%%%%%%%%% ------------ next page ------------  %%%%%%%%%%%%
\newpage
\pagestyle{plain}

\addcontentsline{toc}{chapter}{Acknowledgment}

\begin{flushleft}
{\huge{\bf Acknowledgment}}\\
\vspace{1cm}
Upon proceed with this research, we thank deeply in Tokyo Denki University Future Science, Information and Media Department of Masayuki Iwai professor who have cooperation started in all aspects of the research guidance.
\par
Finally, was both the day-to-day research, Tokyo Denki University Future Science, Information and Media Department of ubiquitous networking laboratory of friends, people I have now to me over the encouragement words of when you blow a Windows little opportunity to touch another , it represents the heartfelt gratitude to four years of families who have supported the college life.

\vspace{3cm}
\begin{flushright}
March 31, 2016\\
Hiroto Takahashi\\
\end{flushright}
\end{flushleft}



%%%%%%%%%%%% ------------ next page ------------  %%%%%%%%%%%%
\newpage



\addcontentsline{toc}{chapter}{Off-campus Conference presentation}

\begin{flushleft}
{\huge{\bf Off-campus Conference presentation}}\\
\vspace{1cm}
\begin{enumerate}

\item \underline{\CID{8705}橋洸人},岩井 将行, "即興的な多人数チームプレイが可能なブラウザネットワーキングゲーム基盤", 情報処理学会 エンタテインメントコンピューティング研究会(SIG-EC).2015年10月.

\item \underline{\CID{8705}橋洸人},岩井将行,“東京エリアストレスー都市エリア毎の感情可視化ツール”,CSISi 第12回公開シンポジウム アーバンデータチャレンジ2015,9月28日.
% \url{http://aigid.jp/?p=1248}

\end{enumerate}
\end{flushleft}

%%%%%%%%%%%% ------------ next page ------------  %%%%%%%%%%%%
\newpage

\renewcommand{\bibname}{References}
\addcontentsline{toc}{chapter}{References}

\begin{thebibliography}{2}

    % MLA format
    \bibitem{mobilesencing}
        重田航平,青木俊介,劉広文,岩井将行,瀬崎薫,“モバイル端末を用いたユーザ参加型環境センシングにおける誤計測地点の検知・修正手法”,マルチメディア,分散,協調とモバイル (DICOMO2013) シンポジウム,セッション 2A-3(2013).
         \url{https://ipsj.ixsq.nii.ac.jp/ej/index.php?action=pages_view_main&active_action=repository_action_common_download&item_id=97168&item_no=1&attribute_id=1&file_no=1&page_id=13&block_id=8}

    \bibitem{socialshare}
         三浦麻子,鳥海不二夫,小森政嗣,松村真宏,平石界,“ソーシャルメディアにおける災害情報の伝播と感情: 東日本大震災に際する事例”,人工知能学会論文誌 0 (2016).
         \url{http://ci.nii.ac.jp/naid/40020080340}

    % ------ twitter vis -------
    \bibitem{twitterlocalevent}
        渡辺 一史, 大知 正直, 岡部 誠, 尾内 理紀夫:
        Twitterを用いた実世界ローカルイベント検出
        \url{http://rit.rakuten.co.jp/conf/rrds4/papers/RRDS4-030.pdf}.

    \bibitem{webpage_fujisawa}
        ふじさわ江の島花火大会(閲覧日: 2016年1月22日):
        \url{http://www.fujisawa-kanko.jp/event/fujisawahanabi.html}.

    \bibitem{webpage_googlemapapi}
        Google Maps API  |  Google Developers(閲覧日: 2016年1月22日):
        \url{https://developers.google.com/maps/?hl=ja}.

    \bibitem{webpage_kanda}
        平成27年度 神田祭/ご遷座四百年奉祝大祭の年(閲覧日: 2016年1月22日):
        \url{http://www.kandamyoujin.or.jp/kandamatsuri/}.

    \bibitem{webpage_nkfes}
        RHYMESTER presents 野外音楽フェスティバル 人間交差点 2016(閲覧日: 2016年1月22日):
        \url{http://www.nkfes.com/}.

    % ------ zuptoon ------
    \bibitem{websocket_webgl}
        Pimentel, Victoria, and Bradford G. Nickerson. "Communicating and displaying real-time data with WebSocket." Internet Computing, IEEE 16.4 (2012): 45-53.
        \url{http://ieeexplore.ieee.org/xpl/login.jsp?tp=&arnumber=6197172&url=http://ieeexplore.ieee.org/xpls/abs_all.jsp?arnumber=6197172}
    \bibitem{websocket_desktop}
        鈴木啓真, and 兼子正勝. "WebSocket を用いたリアルタイムな Web デスクトップ共有." 情報処理学会第 77 回全国大会 1 (2015): 02.
        \url{http://ieeexplore.ieee.org/xpl/login.jsp?tp=&arnumber=6197172&url=http://ieeexplore.ieee.org/xpls/abs_all.jsp?arnumber=6197172}
    \bibitem{qrcoderesearch}
        「二次元コード(QR コード)の使用」に関するアンケート結果(DO HOUSE)
        \url{http://www.dohouse.co.jp/news/research/20140717/}
    \bibitem{smartphone_share}
        総務省|平成24年版 情報通信白書
        \url{http://www.soumu.go.jp/johotsusintokei/whitepaper/ja/h24/html/nc122110.html}
    \bibitem{qrcode}
        QRコードドットコム|株式会社デンソーウェーブ
        \url{http://www.qrcode.com/}
    \bibitem{websocket}
        WebSockets
        \url{https://ajf.me/websocket/}
    \bibitem{inproceedings1}
        小久江 卓哉、中村 貴洋、宮下 芳明:
        WebSocketを用いたWebブラウザ間P2P通信の実現とその応用に関する研究.
        \url{http://ci.nii.ac.jp/naid/110008675481}.
    \bibitem{inproceedings2}
        中村智之、金子晃介、岡田義広:
        携帯端末をデータ放送コンテンツの直観的な入力装置として利用可能とするフレームワークの提案.
        \url{http://ci.nii.ac.jp/naid/110009784022}.
    \bibitem{inproceedings3}
        坂井成道, 峰松美佳, 会津宏幸:
        HTML5 構成変換技術を用いた複数端末への Web ページ分割表示システム
        \url{http://www.toshiba.co.jp/tech/review/2013/12/68_12pdf/f01.pdf}.
    \bibitem{webpagesocketio}
        Socket.IO(閲覧日: 2016年1月22日):
        \url{http://socket.io/}.
    \bibitem{webpageenchatjs}
        enchant.js - A simple JavaScript framework for creating games and apps.(閲覧日: 2016年1月22日):
        \url{http://enchantjs.com/ja/}.

    % ------ trend ------
    \bibitem{socialevent}
        渡辺大貴, and 相場亮. "Twitter を用いた開催中のソーシャルイベントの状況把握に関する研究." 情報処理学会第 77 回全国大会 2 (2015): 05.
        \url{https://ipsj.ixsq.nii.ac.jp/ej/index.php?action=pages_view_main&active_action=repository_action_common_download&item_id=144025&item_no=1&attribute_id=1&file_no=1&page_id=13&block_id=8}

    \bibitem{trendtrigger}
        Zubiaga, Arkaitz, et al. "Real‐time classification of Twitter trends." Journal of the Association for Information Science and Technology 66.3 (2015): 462-473.
        \url{http://onlinelibrary.wiley.com/doi/10.1002/asi.23186/full}

    \bibitem{tdu_trend}
        電大トレンド君 ver2.99(@TDU\_Trend)さん | Twitter
        \url{https://twitter.com/TDU_Trend}

    \bibitem{trend_elzup_com}
        電大トレンド君 on Web
        \url{http://trend.elzup.com/}

    % ------- filter -------

    \bibitem{gps_wrong}
    GPSによる測定値と誤差要因 久保信明
        \url{http://www.denshi.e.kaiyodai.ac.jp/jp/assets/files/pdf/content/201004.pdf}

    % ------- cwm -------
    \bibitem{usersencing}
        Lane, Nicholas D., et al. "A survey of mobile phone sensing." Communications Magazine, IEEE 48.9 (2010): 140-150.
        \url{http://ieeexplore.ieee.org/xpl/login.jsp?tp=&arnumber=5560598&url=http://ieeexplore.ieee.org/xpls/abs_all.jsp?arnumber=5560598}

    \bibitem{aban}
        高橋洸人,岩井将行,“東京エリアストレスー都市エリア毎の感情可視化ツール”,CSISi 第12回公開シンポジウム アーバンデータチャレンジ2015
        \url{http://aigid.jp/?p=1248}

\end{thebibliography}
