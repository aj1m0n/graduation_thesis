%%%%%%%%%%%% ------------ next page ------------  %%%%%%%%%%%%
\newpage

\begin{flushleft}
{\huge{\bf 謝辞}}\\
\vspace{1cm}
<\ >本研究を進めるにあたり,研究指導をはじめあらゆる面でご協力して下さった東京電機大学未来科学部情報メディア学科岩井将行教授に深く感謝致します.
\par
<\ >最後に研究の日々を共にした,東京電機大学未来科学部情報メディア学科ユビキタスネットワーキング研究室の友人達に感謝の意を表します.\\

\vspace{3cm}
\begin{flushright}
2016年3月31日\\
高橋 洸人\\
\end{flushright}
\end{flushleft}




%%%%%%%%%%%% ------------ next page ------------  %%%%%%%%%%%%
\newpage



\begin{flushleft}
{\huge{\bf 学外発表}}\\
\vspace{1cm}
\begin{enumerate}
	
\item \underline{高橋 洸人},岩井 将行, "即興的な多人数チームプレイが可能なブラウザネットワーキングゲーム基盤", 情報処理学会 エンタテインメントコンピューティング研究会(SIG-EC). 2015年10月.

\end{enumerate}
\end{flushleft}

%%%%%%%%%%%% ------------ next page ------------  %%%%%%%%%%%%
\newpage

\renewcommand{\bibname}{参考文献}

\begin{thebibliography}{1}
    \bibitem{inproceedings1}
        小久江 卓哉,中村 貴洋,宮下 芳明:
        WebSocketを用いたWebブラウザ間P2P通信の実現とその応用に関する研究.
        \url{http://ci.nii.ac.jp/naid/110008675481}.

    \bibitem{inproceedings2}
        中村智之,金子晃介,岡田義広:
        携帯端末をデータ放送コンテンツの直観的な入力装置として利用可能とするフレームワークの提案.
        \url{http://ci.nii.ac.jp/naid/110009784022}.

    \bibitem{inproceedings3}
        坂井成道, 峰松美佳, 会津宏幸:
        HTML5 構成変換技術を用いた複数端末への Web ページ分割表示システム
        \url{http://www.toshiba.co.jp/tech/review/2013/12/68_12pdf/f01.pdf}.

    \bibitem{inproceedings4}
        渡辺 一史, 大知 正直, 岡部 誠, 尾内 理紀夫:
        Twitterを用いた実世界ローカルイベント検
        \url{http://rit.rakuten.co.jp/conf/rrds4/papers/RRDS4-030.pdf}.

    \bibitem{webpagesocketio}
        Socket.IO:
        \url{http://socket.io/}.

    \bibitem{webpageenchatjs}
        enchant.js - A simple JavaScript framework for creating games and apps.:
        \url{http://enchantjs.com/ja/}.

    \bibitem{webpagefujisawa}
        ふじさわ江の島花火大会:
        \url{http://www.fujisawa-kanko.jp/event/fujisawahanabi.html}.

\end{thebibliography}
