%%%%%%%%%%%% ------------ chapter header ------------  %%%%%%%%%%%%
\chapterhead
% chapter title
{Twitter のローカルトレンドの抽出}
% chapter abst
{本章では特定のクラスタ内でのリアルタイムなトレンド解析手法の検討について述べる}

%%%%%%%%%%%% ------------ next page ------------  %%%%%%%%%%%%

\section{背景と関連研究}

% 概要 % 目的
社会には様々な組織やグループがあり,その内部でのニュースやトレンドが存在する.また,Twittr はリアルタイムな情報や進行中のイベント取得のためのツールとして活発に使用されている.
関連研究として,ツイート解析によるイベントの混雑状況把握の提案\cite{socialevent}や,トレンドの予測やトリガーとなる事柄の分析手法の提案\cite{trendtrigger}などがされている.
今回は本学学生によるツイートの解析によりローカルなトレンドキーワードの抽出を試みた.

\section{システム概要}
サンプルとして,自己申請されたユーザ及び手動での登録を行った,本学学生の Twitter ユーザ合計約 600 ユーザのツイートを対象に集計し,直近の時間帯に流行しているキーワードの抽出をするアプリケーションを実装した.1 時間毎に統計を行い,上位キーワードをトレンドとして Twitter の bot\footnote{インターネット・ボットとして動いている Twitter アカウント} から投稿をする\figref{twittterbot}\cite{tdu_trend}.

\section{アルゴリズム}
トレンドの評価には\figref{trendana}に示す独自のアルゴリズムを使用した.まず単純にキーワードの出現頻度を計算し,ユーザの人数によりポイントの調整を行い,最近一日間の記録したトレンドと累積した単語のポイントをマイナス評価としてノイズの除去を行う.結果の上位 6 件をツイートし,ポイントは次の評価のためにログと累積に分けてデータベースに記録を残す.

\myfig[width=8cm]{trendbot.png}{実験で作成した Twitter bot アカウント}{trendbot}

トレンドのコンテンツとしての配信についても考察し以下に示す複数の機能を実装した.
\begin{itemize}
    \item 一日や一週間のトレンドの配信
    \item リプライによるキーワード辞書の追加機能
    \item 連続でのトレンド入りの表示
\end{itemize}

%  トレンのド処理フロー
\myfig{trendanaw.png}{トレンドの解析のフロー}{trendana}

% 限界
% 解明点
今回の手法でトレンドの抽出を実験した結果,\figref{trendsample0}で示すように実際に学内で突発的に発生し Twitter 上で話題になった情報を抽出することができた.
Twitter 公式でトレンドとして掲示される単語に重複する結果も多く見られた\figref{trendsample1}.また,くだけた文章の分かち書き精度の甘さから特殊な固有名詞などが抽出されにくく,反対に確実に抽出されるハッシュタグのキーワードはポイントが高く偏る傾向があり改善点である.

% 意義
% 展望
bot での配信では表現が限られるため,\figref{trendweb}のように web サイトの実装も行った\cite{trend_elzup_com}.
クラスタ内だけでなくローカルの特徴的なキーワードに重きをおく評価の導入,感情の評価による統計,Web サイトでの結果表示などの発展が考えられる.


\myfigtwo{trendsample0.png}{学内で発生した突発的な情報を含む例}{trendsample0}
{trendsample1.png}{Twitter 公式のトレンドと重複した例}{trendsample1}
\myfig{trendweb.png}{トレンド集計 Web ページ}{trendweb}


\newpage
