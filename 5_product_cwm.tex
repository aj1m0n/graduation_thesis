%%%%%%%%%%%% ------------ chapter header ------------  %%%%%%%%%%%%
\chapterhead
% chapter title
{モバイル端末センシングサーバAPI}

% chapter abstract
{本章では,モバイル端末に付属のセンサで収集したデータのログを管理するためのサーバアプリケーションの作成について述べる}


%%%%%%%%%%%% ------------ next page ------------  %%%%%%%%%%%%

\section{背景と関連研究}
様々なセンサが付属するスマートフォンの普及を背景に,一般ユーザのスマートフォンを用いたセンシングの実現が期待されている\cite{usersencing}.
今回,複数の研究やプロジェクトでの使用と複数のユーザによるセンサ情報管理を目的とし,ユーザ参加型センシングを実現に必要であるサーバアプリケーション,Web API の作成と Web 画面での管理アプリケーションを作成した.

\section{システム概要}
サーバに保存するデータ構造を\figref{datastructurew}のように定義した.今回は Project 別の管理を目的とし,Project と User は一対多とした.
モバイル端末側のクライアントアプリケーションでは,「プロジェクトのユーザを作成し発行」と「ユーザのセンサ情報を追加」を行う 2つを用いることで単純なセンシングアプリケーションを作成する事ができる.
管理画面はシンプルな UI にし,ユーザ毎やプロジェクト単位でのデータエクスポートを管理画面から行えるようにした\figref{manageproject}\figref{manageuser}.収集したデータの分析のために CSV と KML のフォーマットでのエクスポート機能を実装した.KML は Google Earth や Google Maps でサポートされていて分析に有効である.Google Earth を用いた出力として\figref{kmlroute}と高度情報を表示した\figref{kmlheight}を示す.

% データ構造 Project-User-Logs
\myfig{datastructurew.png}{サーバで扱うデータ構造}{datastructurew}

% 管理画面
\myfig{manageproject.png}{プロジェクト一覧の管理画面}{manageproject}
\myfig{manageuser.png}{データのダウンロードなどを行える User の管理画面}{manageuser}
\vspace{-1cm}

%  Google Earth
\myfigtwo{kmlroute.png}{Google Earth による可視化と分析}{kmlroute}
{kmlheight.png}{高度情報の可視化}{kmlheight}


% 限界
% 解明点
% 意義
% 展望

\newpage
