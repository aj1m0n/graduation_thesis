%%%%%%%%%%%% ------------ 2 page ------------  %%%%%%%%%%%%
\newpage
\pagestyle{plain}
\pagenumbering{roman}
\begin{center}
\LARGE{卒業論文要旨 \hspace{10mm} 2015年度(平成27年度)}\\

\vspace{10mm}

\LARGE{即時性の伴うイベントを\\可視化・共有するWebサービス群}\\
\end{center}

\begin{center}
概要\\
\end{center}

Twitter に代表されるマイクロブログの広まりやスマートフォンの普及により,ソーシャルメディアに人々の自発的で自然な反応が多く含まれていることが可能になった.本研究においても,リアルタイムの個別ユーザからの実世界のイベントに関する反応を入手し,トレンド分析やイベントの整理,共有などを行う様々なサービスを最新のサーバ技術を用いて構築した.リアルタイムな情報共有に注目し,複数のアプリケーションの作成を通し共通基盤を構築したことについて述べる.

\begin{flushleft}キーワード:\\
\end{flushleft}
{\underline{ブラウザネットワーキング},\underline{マイクロブログ},\underline{人流}}


\begin{center}
\vspace{10mm}
\begin{flushright}\large 東京電機大学院未来科学研究科情報メディア学専攻\\
\LARGE \CID{8705}橋 洸人\\
\end{flushright}
\end{center}
