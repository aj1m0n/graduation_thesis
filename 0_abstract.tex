%%%%%%%%%%%% ------------ 2 page ------------  %%%%%%%%%%%%
\newpage
\pagestyle{plain}
\pagenumbering{roman}
\begin{center}
\LARGE{修士論文要旨 \hspace{10mm} 2021年度(令和3年度)}\\

\vspace{10mm}

% \LARGE{}\\
\end{center}

\begin{center}
概要\\
\end{center}
交通工学はエッジセンサから得られたデータを活用することで,
可能性が大きく広がる分野である.以前より,速度違反検知,ナンバープレート検知,渋滞検知などは日本のみならず先進国で導入されている.
それらのセンサは大規模かつ高額になることが多く,交通量の多い交差点や高速道路に限定して設置されていることが多い.
そのため,これらのセンサを活用した交通最適化を行っている事例は筆者の知る限り無い.
また,住宅街や2車線道路などでは,これらのセンサの設置がまだ進んでいないのが現状である.
これらの道路での渋滞は信号機の制御の最適化を行うことで解消すると考えられており,
本研究では次世代の動的な信号機の開発を行なっている.
その中で,センサとして監視カメラと画像処理で車両の状態推定を行うエッジセンサを開発した.
これにより今まで取ることができなかったその地点の詳細な車両のデータの取得が可能になった.


\begin{flushleft}キーワード:\\
エッジセンサ, sort, リアルタイム,状態推定
\end{flushleft}


\begin{center}
\vspace{10mm}
\begin{flushright}\large 東京電機大学院未来科学研究科情報メディア学専攻\\
\LARGE 安齋 凌介\\
\end{flushright}
\end{center}
