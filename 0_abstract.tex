%%%%%%%%%%%% ------------ 2 page ------------  %%%%%%%%%%%%
\newpage
\pagestyle{plain}
\pagenumbering{roman}
\begin{center}
\LARGE{修士論文要旨 \hspace{10mm} 2021年度(令和3年度)}\\

\vspace{10mm}

% \LARGE{}\\
\end{center}

\begin{center}
概要\\
交通工学はエッジセンサから得られたデータを活用することで,
可能性が大きく広がる分野である.速度違反検知,ナンバープレート検知,渋滞検知などは日本のみならず先進国で導入されている.
しかしながら,それらのデータを交通の最適化に活用している例はまだ少ない.
これは,データの質の低さ,ラベルの欠如,データの利権などが障壁になっていることが考えらえる.
またそれらのセンサは大規模かつ高額になることが多く,交通量の多い交差点や高速道路に限定して設置されていることが多い.
したがって,我々が普段使う道路や交差点ではこれらの導入は遅れていることが現状である.そこで,我々はオープンソースデータと安価なエッジセンサを活用し,
交差点付近の車両の状態を推定することができるデバイスと,信号機を最適化するシステムを開発した.
これにより今まで取ることができなかったその地点の詳細な車両のデータの取得と動的な信号機の信号操作を可能にした.
\end{center}


\begin{flushleft}キーワード:\\
yolo, sort, リアルタイム,状態推定
\end{flushleft}
% {\underline{ブラウザネットワーキング},\underline{マイクロブログ},\underlinse{人流}}


\begin{center}
\vspace{10mm}
\begin{flushright}\large 東京電機大学院未来科学研究科情報メディア学専攻\\
\LARGE 安齋 凌介\\
\end{flushright}
\end{center}
