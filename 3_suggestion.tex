%%%%%%%%%%%% ------------ chapter header ------------  %%%%%%%%%%%%
\chapterhead
% chapter title
{提案手法}
% chapter abst
{本章では,提案手法について説明する.}

%信号の全体像(白畑さん,伊藤先生)の部分も書いく
%ハンガリアンアルゴリズム
%フローを記入

%%%%%%%%%%%% ------------ next page ------------  %%%%%%%%%%%%

\section{提案システム}%----------
本研究の目的は1章でも述べたように,リアルタイムで車両の状態を推定することである.%自立分散型信号機に必要な要件が必要
認識する項目は,車両の位置情報,車両の速度,(車両の種類,交差点内の車両の右左折待ち,複数のカメラ間での車両のマッチング,オクルージョンの解決)である.%フィードバック
図 のように各信号機に対応する車道を撮ることができる監視カメラから得られた動画を元に各処理を行う.%kai
本研究では,エッジセンサとしてJetson Xavier NX\cite{jetson}を用いており,動画の取り込みから
オブジェクト認識,追跡までを1台のセンサで行えるようにした.

\subsection{本研究の要件}
提案手法を述べる前に本研究の要件について整理する.
本研究では,自立分散型信号の開発に当たって必要である車両の状態推定を行うためのセンサの開発,検証を行う.
自立分散型信号機のセンサとして以下の要件を満たす必要がある.

\begin{enumerate}
	\item 車両の位置の推定
	\item 車両の右左折待ちの推定
	\item 車両の再認識
\end{enumerate}


\subsection{車両の位置推定}
車両の位置を推定する際に監視カメラから得られた動画内にある車両をオブジェクトとして認識する必要がある.
そこで,YOLOv4\cite{bochkovskiy2020yolov4}を用いて車両を認識する.
ただし,オブジェクトは次フレームでは別のオブジェクトとして認識されてしまうので,
カルマンフィルタを用いて前フレームで認識したオブジェクトを次フレームでも認識できるようにする.
これにより,カメラ画像内のオブジェクトの位置がpixelのXY座標で表すことができるよになる.
この座標を射影変換し,実際の地図に対応させる.射影変換は道路の角を4点決め,地図でもそれに対応した4点を決める.
この点同士からホモグラフィ行列を求めることで,カメラ画像内のオブジェクトの座標がリアルタイムで地図上に反映される.

%kai

\subsection{車両の速度}
車両の位置推定を行った後,地図上で速度を計測する,

\section{Re-id}
4台のカメラで捉えた同一車両のマッチングを行う.
1台のカメラから捉えた車両の特徴は全てのエッジセンサに共有される.
共有される特徴は,ID,車種,移動軌跡となる.


\subsection{車両の種類}
車両の種類の特定は,YOLOv4の◯◯を用いた.
  
\subsection{交差点内の右左折待ち}
交差点内の右左折待ちは,時間と車両の位置,ウィンカーの有無で判定を行う.


\subsection{オクルージョンの解決}
車両が前方の車両に隠れて見えなくなってしまことがある.このオクルージョンを解決するために
道路の範囲を指定し,オクルージョンが発生する前に認識した車両は前方の車に隠れていると判定するようにした.


\newpage
