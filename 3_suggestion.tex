%%%%%%%%%%%% ------------ chapter header ------------  %%%%%%%%%%%%
\chapterhead
% chapter title
{提案手法}
% chapter abst
{本章では,提案手法について説明する.}


%%%%%%%%%%%% ------------ next page ------------  %%%%%%%%%%%%

\section{提案システム}%----------
本研究では,アルゴリズムにSORT,安価なエッジセンサとしてJetson Xavier NX\cite{jetson},
オブジェクト検出機としてYolov4\cite{bochkovskiy2020yolov4}を用いて,
交差点付近の車両を追跡するシステムを開発した.それらを各信号機に設置し,
% AMQP(Advanced Message Queuing Protocol)の
各センサと各信号機を制御するRaspberryPi4を接続している.

このシステムを開発する際の問題点として,

\begin{itemize}
  \item[(1)]安価であるが故に十分な性能を発揮することが難しいこと
  \item[(2)]既存のデバイスで得られたデータを活用することが難しいこと
  \item[(3)]環境的に通信の信頼性を担保することが難しいこと
\end{itemize}
がある.(1)はJetson Xaviar NXを用いて,検証を行う.さらに,SORTは実装が軽量なためリアルタイム処理性能として非常に優れており,本研究ではこれを用いることとした.
(2)はThe Microsoft Common Objects in COntext (MS COCO) dataset\cite{lin2014microsoft}
を用いて学習されたYOLOv4を使用することで解決する.
(3)は金融系で用いられことを目的としたAMQPのプロトコルを使用することで信頼性を担保する.

\subsection{概要}
提案する車両の追跡システムの概要は図に示す.
提案するシステムはエッジセンサで車両の追跡を行い,クライアント側で座標データを受け取って,地図上に描画する.
エッジセンサ側では,検出器にYOLOv4を用いて,SORTアルゴリズム\cite{wojke2017simple}を用いて車両の追跡を行う.
クライアント側では,ホモグラフィ行列を求め,Google Maps API\cite{googlemap}用いて地図に車両の位置情報を描画を行う.
今回の実験での信号機の制御は,センサから受け取った車両の位置情報から最適な信号制御を行う.




\newpage
