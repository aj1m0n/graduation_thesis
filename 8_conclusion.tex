%%%%%%%%%%%% ------------ chapter header ------------  %%%%%%%%%%%%
\chapterhead
% chapter number
{8}
% chapter title
{結論}
% chapter abst
{本章では,本研究で得られた結果をもとに,結論を述べる.}
%%%%%%%%%%%% ------------ next page ------------  %%%%%%%%%%%%

\section{まとめ}
本研究では,広域無線マルチホップセンサネットワークの構築例として,TScan:微気象センサネットワークを報告し,センサネットワークの長期間運用するための課題を述べた.また,センサノード長期運用における課題を解決するため,ニューラルネットワークを用いてセンサデバイスやセンサネットワークの故障解析を行った.ニューラルネットワークの教師データとして,TScan:微気象センサネットワークで取得したセンシングデータを使用し,バックプロパゲーション学習させることにより,長期稼働を行っているセンサネットワークに特化した故障解析が可能になる.学習済みの故障解析ニューラルネットワークを使用し,検証実験を行い,センサ部の故障や停電・バッテリ切れを高精度で解析することが可能であることが示された.ニューラルネットワークの故障解析には,次のような入力データが故障解析結果に大きな影響を与えることがわかった.
\begin{itemize}
\item 気温データ,湿度データ
\end{itemize}
 検証実験によりセンサ故障をはじめとしたセンサデバイスの故障解析するために非常に重要なデータとなることが示された.
\begin{itemize}
\item 送信カウンタ,LQI(電波強度)
\end{itemize}
 検証実験により電波障害と下流ノード故障を解析するために非常に重要なデータであることが示された.しかし,LQIの値は不安定であり,解析精度に影響を与える.そのため,LQIの値の閾値を細かくすることや過去のLQIの値の平均をとることが必要であることがわかった.\\

ニューラルネットワークを用いた故障解析は,精度が低い出力結果もあったが,検証実験ではセンシングデータから主原因をすべて解析することができた.本研究により,ニューラルネットワークによる故障解析は,長期運用を行うセンサデバイスや広域のセンサネットワークをはじめ様々なサービスの故障解析として適しているといえる.

%%%%%%%%%%%% ------------ next page ------------  %%%%%%%%%%%%
\newpage

\section{今後の課題}
今後,センサネットワークの構築・故障解析を行っていく上で,達成すべき課題をセンサネットワーク面,故障解析面から述べる.
