%%%%%%%%%%%% ------------ chapter header ------------  %%%%%%%%%%%%
\chapterhead
% chapter title
{おわりに}
% chapter abst
{本章では,本研究で得られた結果をもとに,結論を述べる.}
%%%%%%%%%%%% ------------ next page ------------  %%%%%%%%%%%%

\section{まとめ}
Twitterに代表されるマイクロブログの広がりやスマートフォンの普及により,それらをインフラとして人々の生活を変えより利便性を向上にしていくにはサーバサイドの技術が必要不可欠である.本研究においても,リアルタイムの個別ユーザからの実世界のイベントに関する反応を入手し,トレンド分析やイベントの整理,共有などを行う様々なサービスを最新のサーバ技術を用いて構築した.

\section{今後の課題}
我々は本稿で紹介した,即時スマートフォン参加アプリケーション,トレンド配信 bot,参加型センシングサーバ API を拡張し実運用やユーザ評価などを通して Web サービス群の最新の情報可視化共有基盤をさらにオープンソースなどを通じて広めていく.
