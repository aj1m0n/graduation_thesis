%%%%%%%%%%%% ------------ chapter header ------------  %%%%%%%%%%%%
\chapterhead
% chapter title
{関連研究}
% chapter abstraction
{本章では,本研究における関連研究について述べる.}

\section{関連研究}

\subsection{先行研究}
自立分散や動的な信号制御の先行研究として林ら\cite{hayashi}小中ら\cite{konaka}白井ら\cite{shirai}などがある.

林らは,既存の信号制御である集中制御方式では,交通状況に応じた制御がおよそ10分掛かることを指摘し,自立分散型信号制御システムの有効性と交通シミュレーション
SUMOO(Simulation ofUrban MObility) を用いて,大型と普通車両などの車種別を考慮したモデルを提案した.
小中らも,自立分散型信号機による交通渋滞の緩和に有用である可能性を示唆した.
さらに,白井らも分散型リアルタイム信号機制御システムとして,高性能なマシンを用いたセンサを使い,実空間をシミュレーション上反映し実際の信号機の制御を
行うシステムの提案を行った.

しかしこれらの自立分散型信号機の研究は全てシミュレーション上で行ったものであり,その他の研究でも実空間での実験と検討を行った研究は筆者の知る限り無い.
従って,自立分散型信号機の研究において,実空間上でセンサを用いた動的な信号制御を行い,これらの研究の自立分散型信号機の研究において実証が急務である.
%kai (フローチャート)

\subsection{ITSについて}
IoT(Internet of Things)の普及に伴い,さまざまな場所でのセンシングやデータの収集が可能となった.
交通工学の分野でも例外ではなく,古くは車両データを扱った技術として速度違反自動取締装置(オービス)が挙げられる.wakatime
また,近年ではITS(Intelligent Transport Systems:高度道路交通システム)
の観点からさらに高度で詳細な交通データを扱う研究やプロジェクトが注目されている.
IEEE/CVF Conference on Computer Vision and Pattern Recognition (CVPR) Workshopsでは
AI City Challenge\cite{Naphade_2020_CVPR_Workshops}と称し,ITSに関するワークショップを毎年開催している.
ここでは,カメラによる車両の識別,スピード検知,ナンバープレートの読み込みによるマルチカメラでの車両の再識別などのタスクが用意されており,
国際的に見てもITSが注目されていることがわかる.

\subsection{日本の交通事情}
現在,センサ類が道路に対して取り付けられる実例は都会を中心に増えつつあるものの,高速道路や交通量の多い幹線道路に限定していることが多い.
取り付けられている道路においても,それらのデータはスピード違反検知や盗難車両特定,渋滞の検知などとして用いられていることが多く,
交通の最適化という観点からデータを活用している事例は少ない.
そもそも最先端技術を一般道路に活用しようとする働きかけがまだまだ少ないのも現状である.
海外では,中国を筆頭にITSが急速に成長している.%kai(関連記事,研究,添付)
従って,次世代のインフラについて考える機会が増えつつある.

\subsection{信号機について}
インフラの中でも重要な要素の一つに信号機がある.信号機は車両用信号機と歩行者用信号,路面電車用信号機の3つに分けることができる.
本研究では車両用信号機のみについて考える.
現行の車両用信号機の制御は,「制御対象の信号交差点間の関連性」と「信号制御パラメータの設定方式」の2つに分類することができる.
本研究では,動的な信号制御を扱うことから,白畑ら\cite{shirahata}を参考に後者の制御について考える.
制御方法は,定周期制御と交通感応制御の2つに分けることができる.定周期制御は,時間帯に応じてあらかじめ信号制御のパラメータが設定されているもので,
交通感応制御は,車両検知器などを用いて制御する方式である.

\subsection{画像処理}

カメラ画像から複数のオブジェクトを検出し追跡するタスクをコンピュータビジョンや機械学習分野ではMOT(Multi Object Tracking)と呼ぶ.
MOTでは,リアルタイムか否かで大きく二つにタスクを分けることができる.
リアルタイムではない処理としてTNT(TrackletNet Tracker)は,2次元のカメラ映像からオブジェクトを検出し,オブジェクトの軌跡を深層学習
を用いて推定していく方法である.\cite{wang2019exploit} \cite{tang2018single} TNTは,カメラ映像内で複雑かつ大量のオブジェクトを追跡することに向いているものの,カメラ映像を全て読み込んで処理するため,リアルタイム性
はなく,本研究では用いることができない.
一方でリアルタイムでの処理として,SORT(Simple Online and Realtime Tracking)がある.\cite{bewley2016simple} \cite{wojke2017simple}
これは,検出されたオブジェクトの座標をカルマンフィルタ\cite{kalman1960new}を用いて,リアルタイムでのオブジェクトの追跡を可能としている.
前述で述べたTNTと比べると複雑なMOTを行うことは難しいものの,ある程度規則性を持った車両などのMOTはリアルタイムで処理することが可能である.


\newpage

