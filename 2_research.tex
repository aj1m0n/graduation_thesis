%%%%%%%%%%%% ------------ chapter header ------------  %%%%%%%%%%%%
\chapterhead
% chapter title
{関連研究}
% chapter abstraction
{本章では,}

\section{関連研究}
IoT(Internet of Things)の普及に伴い,さまざまな場所でのセンシングやデータの収集が可能となった.
交通工学の分野でも例外ではなく,古くは車両データを扱った技術として速度違反自動取締装置(オービス)が挙げられる.
また,近年ではITS(Intelligent Transport Systems:高度道路交通システム)
の観点からさらに高度で詳細な交通データを扱う研究やプロジェクトが注目されている.
IEEE/CVF Conference on Computer Vision and Pattern Recognition (CVPR) Workshopsでは
AI City Challenge\cite{Naphade_2020_CVPR_Workshops}と称し,ITSに関するワークショップを毎年開催している.
ここでは,カメラによる車両の識別,スピード検知,ナンバープレートの読み込みによるマルチカメラでの車両の再識別などのタスクが用意されており,
国際的に見てもITSが注目されていることがわかる.

しかし,現状としてそれらの装置が一般的な道路や交差点に取り付けられていることは少なく,高速道路や交通量の多い幹線道路に限定し取り付けれていることが多い.
これは,装置自体が高価で大規模化しやすいことが原因であると考えられる.
さらに,日本においてはそれらのデータは,違反や犯罪,渋滞の検知として用いられていることが多いものの,
交通の最適化という観点からデータを活用している事例は少ない.
これは,データの質の低さ,ラベルの欠如,データの利権などが障壁になっていることや
そもそも最先端技術を一般道路に活用しようとする働きかけが少ないことが考えられる.
しかし,働きかけは技術の進歩とともに徐々に増えていくものであり事実,〇〇 %1
やトヨタモビリティなど
次世代のインフラについて考える機会が増えつつある.
したがって,交差点や道路上の車両の状態を推定するには,IoTデバイスの開発を公開されているデータや安価なエッジセンサで行う必要がある.
これらのことから本研究では,2次元のカメラ映像から交差点付近にある複数台の車両の追跡をリアルタイムで行うデバイスの開発を目標とした.

こうしたカメラ画像から複数のオブジェクトを検出し追跡するタスクをコンピュータビジョンや機械学習分野ではMOT(Multi Object Tracking)と呼ぶ.
MOTでは,リアルタイムか否かで大きく二つにタスクを分けることができる.
リアルタイムではない処理としてTNT(TrackletNet Tracker)は,2次元のカメラ映像からオブジェクトを検出し,オブジェクトの軌跡を深層学習
を用いて推定していく方法である.\cite{wang2019exploit} \cite{tang2018single} TNTは,カメラ映像内で複雑かつ大量のオブジェクトを追跡することに向いているものの,カメラ映像を全て読み込んで処理するため,リアルタイム性
はなく,本研究では用いることができない.
一方でリアルタイムでの処理として,SORT(Simple Online and Realtime Tracking)がある.\cite{bewley2016simple} \cite{wojke2017simple}
これは,検出されたオブジェクトの座標をカルマンフィルタ\cite{kalman1960new}を用いて,リアルタイムでのオブジェクトの追跡を可能としている.
前述で述べたTNTと比べると複雑なMOTを行うことは難しいものの,ある程度規則性を持った車両などのMOTはリアルタイムで処理することが可能である.

\newpage
