%%%%%%%%%%%% ------------ next page ------------  %%%%%%%%%%%%
\newpage
\setcounter{chapter}{4}
\setcounter{section}{0}
%\setcounter{subsubsection}{0}

\begin{center}
\vspace{0.5cm}
\huge{\bf 第4章}
\par
\vspace{1cm}
\hrulefill
\par
\vspace{1cm}
\huge{\bf 位置情報付きツイート解析による経路可視化}
\par
\vspace{0.5cm}
\hrulefill
\vspace{1cm}
\par

%\normalsize
\begin{flushleft}
\large{{\bf TODO: 本章では,故障解析に使用するニューラルネットワークと,TScanの適用方法について述べる.}}
\end{flushleft}
\end{center}

\addcontentsline{toc}{chapter}{\protect\numberline{第4章}{位置情報付きツイート解析による経路可視化}}

%%%%%%%%%%%% ------------ next page ------------  %%%%%%%%%%%%
\newpage
\section{概要}
本研究では,微気象センサネットワークの故障解析にニューラルネットワークを利用する.本章では,ニューラルネットワークに用いた故障解析手法について述べる.

%%%%%%%%%%%% ------------ next page ------------  %%%%%%%%%%%%
\newpage

\section{本章のまとめ}
本章では,バックプロパゲーションの概要を述べ,TScan故障解析への適用方法について述べた.また,TScan:微気象センサネットワークから取得されたデータから,教師データ,正常時のデータ,センサ故障時データ,電波障害時データ,停電・バッテリ切れデータ,下流ノード故障時データを閾値を設けてニューラルネットワークの入力データとした.準備した入力データからTScan故障解析検証実験を行った.次章では,検証実験の結果について述べる.
\newpage
