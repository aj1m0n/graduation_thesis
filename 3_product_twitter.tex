%%%%%%%%%%%% ------------ chapter header ------------  %%%%%%%%%%%%
\chapterhead
% chapter number, title
{3}{位置情報付きツイート解析による経路・イベント検知と可視化}
% chapter abst
{本章では,位置情報付きツイートに着目しイベント参加者と思われる人の移動経路, イベント検知と可視化について述べる.}

%%%%%%%%%%%% ------------ next page ------------  %%%%%%%%%%%%

\section{システム概要}
本研究では,位置情報付きツイートに着目しイベント参加者と思われる人の移動経路の可視化とアプリケーションの作成を行った.
また,ツイート内容と位置情報からクラスタリングしイベントの予測と可視化をするWebアプリケーションに述べる.

\subsection{移動経路可視化}
 研究のサンプルとして2014年10月18日に行われた藤沢市の花火大会からツイートを収集し移動経路の可視化を行った
* イベント参加者と思われるユーザ: 半径10km以内, + 「花火」「ふじさわ」,主要駅名でのツイート
* 画像

\subsection{イベント検知}
* 神田祭り
* 2015年5月8日〜5月13日

TODO:

%%%%%%%%%%%% ------------ next page ------------  %%%%%%%%%%%%
\newpage

\section{システム構成}
\subsection{移動経路可視化}
\subsubsection{サンプルの取得}
TODO:
* TwitterAPI を使用した, キーワード指定, 円形の位置情報指定ができる
* 同ユーザの前後のツイート

TODO:

\subsubsection{可視化Webアプリケーション}
TODO:

\subsection{イベント検知}
\subsubsection{ツイート収集}
TODO:
\subsubsection{クラスタリング}
TODO:

\subsubsection{可視化Webアプリケーション}
TODO:

%%%%%%%%%%%% ------------ next page ------------  %%%%%%%%%%%%
\newpage

\section{本章のまとめ}
TODO:
% 限界

% 解明点

% 意義

% 展望

\newpage
